\chapter{Einleitung}
\label{sec:einleitung}
Im ersten Kapitelabschnitt der Bachelorarbeit, wird auf die Motivation und die Zielsetzung eingegangen. Zusätzlich wird ein Überblick über den Aufbau der Arbeit aufgezeigt.

\section{Motivation}
\label{sec:motivation}
Beim Suchen und Finden von Lösungen, ungewöhnlichen Geschäftsideen, Innovationen oder um einzelne Projekte erfolgreicher zu machen, bereichert viele Menschen der Begriff Kreativität. Um die Kreativität zu fördern, braucht es Kreativitätstechniken, die dabei helfen, Ideen zu generieren und Einfälle zu sammeln.\bigskip

Der Klassiker und eine der bekannteste unter allen Kreativitätstechniken ist das klassische Brainstorming. Sie wurde vom Amerikaner Alex Faickney Osborn erfunden und von Charles Hutchison Clark zur Ideenfindung innerhalb von Gruppen weiterentwickelt. 
(vgl. \cite{Ben.o.J.}) \glqq Er benannte das Brainstorming nach der Idee dieser Methode, nämlich using the brain to storm a problem (wörtlich: Das Gehirn verwenden zum Sturm auf ein Problem).\grqq{} \cite{Pas2012}\bigskip

Die Kreativitätstechnik Brainstorming gilt als eine der beliebtesten Methoden zur Ideenfindung und -sammlung neuen Geschäftsideen, Ideen für ein Projekt/Produkt oder auch zu einer vorhandenen bzw. gegebenen Problemstellung.\bigskip

Ziel des Brainstormings ist es, Denkblockaden auf der Suche nach neuen Ideen zu beenden. Diese Kreativitätstechnik wird häufig in Seminaren und Workshops angewendet, um die Gruppenarbeit effektiver und effizienter zu gestalten. Bei einer Brainstorming-Sitzung in einem Workshop kann jeder Teilnehmer auf die Ideen des anderen aufbauen und anknüpfen. Dadurch werden die Teilnehmer gegenseitig durch Ihre Ideen zu neuen Ideen angeregt, wodurch mehr Ergebnisse, als tatsächlich gebraucht, produziert werden.\bigskip

Eine häufig angewendete Methodik für die Ausarbeitung des Brainstormings in den Workshops ist es, sich Karteikarten oder Notizzettel zu nehmen, seine Ideen und Gedanken darauf zu schreiben und an eine Pinnwand (Flipchart, Whiteboard) anzubringen. Haben alle Teilnehmer Ihre Karteikarten an der Pinnwand angebracht, wird anschließend analysiert und darüber diskutiert. Am Ende der Besprechung werden die gesammelten Daten bewertet und anschließend von dem Moderator dokumentiert. Mit herkömmlichen analogen Workshops bedeutet das für den Moderator, dass er die Karteikarten auf der Pinnwand abtippen oder abfotografieren muss, um eine Dokumentation erstellen zu können. Da wir uns heutzutage in einem digitalen Zeitalter befinden und uns dieser neuen Welt nicht mehr entziehen können, gilt es, diesen Wandel als Chance zu begreifen, solche analogen Workshops zu digitalisieren, um dem Moderator eine Möglichkeit anzubieten, die Daten digital zusammenzufassen.

\section{Zielsetzung und Aufbau der Arbeit}
\label{subsec:zielsetzung}
Im Rahmen dieser Bachelorarbeit soll eine dynamische Webanwendung (Workshoppy) zur Durchführung von Workshops in Echtzeit entwickelt werden, die das klassische Brainstorming digitalisieren und effektiver machen soll. Die Webanwendung soll künftig in den Workshops genutzt werden und muss die Funktionen bieten, welche mehrere Personen (Teilnehmer) über Ihre Endgeräte (Smartphone, Laptop oder Tablet) ihre Ideen abgeben können. Dabei werden die eingebrachten Ideen der Teilnehmer in Echtzeit auf einer großen Leinwand (Beamer) präsentiert. Der Moderator soll anschließend die Möglichkeit erhalten, die Ergebnisse digital zusammenzufassen. Die Zusammenfassung soll auch als PDF-Datei exportiert werden können. Bei der Konzeption der Webanwendung ist zu beachten, dass eine benutzerfreundliche Darstellung für die Anwender gewährleistet ist.\bigskip

Die vorliegende Arbeit ist wie folgt aufgebaut: Das Kapitel \textbf{\ref{sec:grundlagen}} stellt vorab ein Überblick über einige grundlegende Begriffe vor. Der Begriff Responsive Webdesign, AJAX-Technologie und Rich Internet Applications (RIA) werden besprochen. Anschließend wird der Thin Client und Thick Client beschrieben. Das Kapitel \textbf{\ref{sec:analyse}} beschäftigt sich zunächst mit dem Stand der Technik. Die Anforderung zur Webanwendung wird dabei analysiert und konzipiert. In diesem Kapitel werden vor allem die funktionale, nicht-funktionale Anforderungen sowie die Muss- und Kann- Anforderungen ermittelt. Aufbauend auf den Ergebnissen der Anforderungsanalyse erfolgt in Kapitel \textbf{\ref{sec:design}} eine ausführliche Beschreibung über den Entwurf der Benutzeroberfläche (GUI) der Webanwendung. Danach wird das Design der GUI entworfen. Im Kapitel \textbf{\ref{sec:implementierung}} wird zunächst die zu verwendenden Webentwicklungswerkzeuge vorgestellt. Anschließend beschäftigt sich dieses Kapitel hauptsächlich mit der Implementierung der Webanwendung. Zum Schluss wird es im Kapitel \textbf{\ref{sec:fazit}} die erarbeiteten Ergebnisse zusammengefasst, sowie Ideen für zukünftigen Erweiterungen der entwickelten Webanwendung diskutiert.
