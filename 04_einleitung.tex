\chapter{Einleitung}
\label{sec:einleitung}
Im ersten Kapitelabschnitt der Bachelorarbeit, wird auf die Motivation und die Zielsetzung eingegangen. Zusätzlich wird ein Überblick über den Aufbau der Arbeit aufgezeigt.

\section{Motivation}
\label{sec:motivation}
Beim Suchen und Finden von Lösungen, ungewöhnlichen Geschäftsideen, Innovationen oder um einzelne Projekte erfolgreicher zu machen, bereichert viele Menschen der Begriff Kreativität. Um die Kreativität zu fördern, braucht es Kreativitätstechniken, die dabei helfen, Ideen zu generieren und Einfälle zu sammeln.
\\

Der Klassiker und eine der beliebtesten unter allen Kreativitätstechniken ist das Brainstorming (Gehirnsturm). Das Brainstorming wurde von Alex Faickney Osborn im Jahr 1939 erfunden und von Charles Hutchison Clark zur Ideenfindung innerhalb von Gruppen weiterentwickelt. Die Kreativitätstechnik Brainstorming gilt als eine der effizienteste Methode zur Ideenfindung und -sammlung neuer Geschäftsideen, Ideen für ein Projekt/Produkt oder auch zu einer vorhandenen bzw. gegebenen Problemstellung. [Clark 1972] 
\\

Ziel des Brainstormings ist es, Denkblockaden auf der Suche nach neuen Ideen zu beenden und innerhalb von kurzer Zeit eine große Menge von Ideen zu generieren. Diese Kreativitätstechnik wird häufig in Seminaren und Workshops angewendet, um die Gruppenarbeit effektiver und effizienter zu gestalten. Bei der Brainstorming-Sitzung in einem Workshop kann jeder Teilnehmer auf die Ideen des anderen aufbauen und anknüpfen (Ideenverknüpfungen). Dadurch werden die Teilnehmer gegenseitig durch Ihre Ideen zu neuen Ideen angeregt, wodurch mehr Ergebnisse, als tatsächlich gebraucht, produziert werden. 
\\

Eine alte Methodik für die Ausarbeitung des Brainstormings war es, sich Karteikarten oder Notizzettel zu nehmen, seine Ideen und Gedanken darauf zu schreiben und an eine Pinnwand (Flipchart, Whiteboard) anzubringen. Haben alle Teilnehmer Ihre Karteikarten an der Pinnwand angebracht, wird anschließend darüber analysiert und diskutiert. Am Ende der Besprechung werden die gesammelten Ideen bewertet. In Zeiten von Digitalisierung ist dieses Vorgehen als überholt zu erachten.

\newpage
\section{Zielsetzung und Aufbau der Arbeit}
\label{subsec:zielsetzung}
Im Rahmen dieser Bachelorarbeit soll eine dynamische Webanwendung (Workshoppy) zur Durchführung von Workshops in Echtzeit entwickeln werden, die das Brainstorming digitalisiert und effektiver machen soll. Die Webanwendung soll künftig in den Workshops genutzt werden und muss die Funktionen haben, dass mehrere Personen (Teilnehmer) über Ihre Endgeräte (Smartphone, Laptop oder Tablet) seine Ideen abgeben können. Dabei werden die eingebrachten Ideen von den Teilnehmer in Echtzeit auf einer großen Leinwand - hier ist es der Beamer - präsentiert. Der Moderator soll anschließend die Möglichkeit erhalten, die Ergebnisse in Kategorien zusammenzufassen. Die Zusammenfassung soll auch als PDF-Datei exportiert werden können. Bei der Konzeption der Webanwendung ist zu beachten, dass eine benutzerfreundliche Darstellung für die Anwender gewährleistet ist.
\\

Die vorliegende Arbeit ist wie folgt aufgebaut: In Kapitel 2 werden zunächst die Anforderungen zur Webanwendung analysiert und konzipiert. In diesem Kapitel werden vor allem die funktionale, nicht-funktionale Anforderungen sowie die Muss- und Kann- Anforderungen ermittelt. Nach der abgeschlossenen Anforderungsanalyse werden anhand dieser im Kapitel 3 die zu verwendenden Webentwicklungswerkzeuge angesprochen. Im Kapitel 4 wird die Konkurrenten recherchiert und analysiert. Aufbauend auf den Ergebnissen der Anforderungsanalyse erfolgt in Kapitel 5 eine ausführliche Beschreibung der Implementierung der Webanwendung. Nach der erfolgreichen Implementierung der Webanwendung wird die Evaluierung in Kapitel 6 durchgeführt. Hier wird vor allem die Benutzeroberfläche der Webanwendung und die Funktionalität der Webanwendung getestet. Im letzten Kapitel werden anschließend die erarbeiteten Ergebnisse zusammengefasst, sowie Ideen für zukünftige Erweiterungen der entwickelten Software diskutiert.
