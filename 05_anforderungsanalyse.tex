\chapter{Anforderungsanalyse und Konzeption}
\label{sec:anforderungsanalyse}
Dieses Kapitel befasst sich mit der Anforderungsanalyse und Konzeption der Webanwendung. Dazu werden zunächst eine allgemeine Struktur festgelegt, wie das Projekt systematisch aufgebaut sein soll.
\\

Anschließend wird der aktuelle Zustand (Ist-Analyse) des Projektes ermittelt. Anhand dieser Ist-Analyse erfolgen die funktionalen und nicht funktionalen Anforderungen an die zu entwickelnde Webanwendung. Am Ende wird die Architektur der Webanwendung erklärt.


\section{Projektstruktur}
\label{sec:projektstruktur}
Nach Definition der DIN (Deutsches Institut für Normung e.v., 2009) 69901-5:2009 ist der Projektstrukturplan die \textit{“ [...] vollständige hierarchische Darstellung aller Elemente (Teilprojekte, Arbeitspakete) der Projektstruktur als Diagramm oder Liste.”}
\\

Auf der oberste Ebene steht das Projekt. Eine Ebene darunter die Teilprojekte oder Teilaufgaben, darunter schließlich die Arbeitspakete. 
Der Projektstrukturplan (Abbildung 2.1) entspricht dem typischen sequentiellen Vorgehensmodell zur Softwareentwicklung einschließlich der Entwicklung der Webanwendung.

\begin{figure}[H]
  \centering  
  \includegraphics[scale=0.3]{img/Projektstrukturplan.jpg}
  \caption{Projektstrukturplan [Quelle: eigene Abbildung]}
  \label{fig:projektstrukturplan}
\end{figure}

\newpage
Eine Analyse von funktionalen und nicht-funktionalen Anforderungen sowie die Muss- und Kann-Anforderungen wird in der ersten Phase untersucht. Anhand dieser Analyse wird die Struktur und ein passendes Layout der Webanwendung erstellt. Der daraus entstehende Entwurf wird technisch in eine Webanwendung umgesetzt und am Ende getestet.

\section{Ist-Analyse}
\label{sec:ist-analyse}



\section{Anforderungsanalyse}
\label{sec:anforderungsanalyse}
Dieses Kapitel umfasst die grundlegende Anforderungen dieser Bachelorarbeit. Die Anforderung wird in funktionale und nicht- funktionale Anforderungen aufgeteilt. 
\\

Eine funktionale Anforderung wird nach der Definition aus dem Buch \cite{Balzert2010} die gewünschte Funktionalität des Systems bzw. eines Produkts beschrieben. 
Die nicht-funktionale Anforderungen sind Anforderungen, die für die Nutzung des Systems wichtig sind. Außerdem wird eine Muss- und Kann-Anforderung formuliert, welche für das Projekt oberste Priorität hat und welche eher zweitrangig ist.

\subsection{Funktionale Anforderungen}
\label{sec:funktionale anforderungen}
\label{sec:funktionale anforderungen}
Aus den Unternehmensanforderungen lassen sich folgende funktionale Anforderungen ableiten.

\subsubsection*{Startseite}
Die Startseite der Webanwendung ist die Eingangsstelle für eine moderierende Person, nachdem diese sich bei der Anwendung angemeldet hat. Er kann auf dieser Seite neue Workshops erstellen, sie bearbeiten sowie löschen. Jeder Workshop hat einen Titel und wird in einer Datenbank gespeichert. Die erstellten Workshops werden nacheinander aufgelistet. Auf der Startseite soll außerdem eine Liste der beendeten Workshops anzeigen. In dieser Liste sind die Workshops, die von der moderierenden Person vorher beendet wurden. Das Datum und die Uhrzeit, an dem die Workshops beendet wurden, soll ebenfalls ausgegeben werden. Außerdem soll jeder beendete Workshop einen Button besitzen, der die Ergebnisse des Workshops anzeigt. 

\subsubsection*{Controller-Seite}
Der ausgewählte Workshop soll zur Controller-Seite führen. Diese Seite beinhaltet unter anderem den Titel vom ausgewählten Workshop und eine Liste der Sessions. Die Sessions können vom Moderator erstellt werden. Er soll sie auch bearbeiten und löschen können. Beim Erstellen einer neuen Session soll neben den Titel auch eine Frage, die behandelt wird, angegeben werden. Die Frage wird als Pflichtfeld gekennzeichnet. Die Sessions werden ebenfalls wie bei den Workshops in einer Datenbank gespeichert. Beim Bearbeiten einer Session soll der Moderator neben Titel- und Fragenänderung auch als Option Kategorien zu dieser Session hinzufügen können.
\\ 

Eine Session versteht sich als eine Sitzung zur Ideenfindung und -sammlung, um Lösungen für eine Problemstellung zu finden. Der Moderator kann zu jedem Workshop mehrere Sessions erstellen.
\\

Jede Session auf der Controller-Seite eines ausgewählten Workshops muss drei Buttons enthalten, die der moderierenden Person folgende Funktionen anbieten:
\begin{itemize}
\item Session starten-Button:
\begin{itemize}
\item um Lösungen und Ideen für eine Problemstellung zu sammeln, muss die Session gestartet werden. Hat die moderierende Person eine Session gestartet, soll automatisch die Präsentation-Seite aufgerufen werden. Während eine Session läuft, sollte dieser Button bei den anderen Sessions deaktiviert sein.
\end{itemize}
\item Eingabe beenden-Button:
\begin{itemize}
\item beendet die Funktion zur Dateneingabe seitens der Teilnehmer. Auf der Präsentation-Seite soll nach dem Betätigen dieses Buttons ein Button zur Erstellen von Kategorien freigeschaltet werden.
\end{itemize}
\item Session beenden-Button:
\begin{itemize}
\item beendet die gestartete Session. Die Ergebnisse sollen anschließend in einer Datenbank gespeichert werden.
\item reaktiviert die zuvor deaktivierten Session starten-Buttons.
\end{itemize}
\end{itemize}

Folgenden Buttons müssen ebenfalls auf der Controller-Seite zur Verfügung stehen:
\begin{itemize}
\item Client-Button:
\begin{itemize}
\item zeigt die Teilnehmer-Seite. Sie stellt den Teilnehmern die Funktionen für Dateneingabe bereit.
\end{itemize}
\item Präsentation-Button:
\begin{itemize}
\item zeigt die Präsentation-Seite. Die Präsentation-Seite wird über dem Beamer angezeigt und präsentiert die eingegebenen Daten von allen Teilnehmern in Echtzeit.
\end{itemize}
\item Ergebnisse-Button:
\begin{itemize}
\item ruft die Ergebnisse-Seite auf. Die Ergebnisse von allen Sessions eines ausgewählten Workshops werden auf der Seite in Form einer Tabelle präsentiert.
\end{itemize}
\item Workshop Beenden-Button:
\begin{itemize}
\item beendet den ausgewählten Workshop und führt den Moderator zu Startseite zurück. Der Workshop soll sich anschließend in der Liste der beendeten Workshops befinden.
\end{itemize}
\end{itemize}

Die Controller-Seite eines ausgewählten Workshops soll außerdem dem Moderator die Funktion anbieten, die es ihm erlaubt, den Teilnehmer eine Einladungsmail zur Teilnahme am Workshop zu senden.

\subsubsection*{Teilnehmer-Seite}
Die Teilnehmer-Seite soll jedem Teilnehmer am Workshop die Funktion zur Dateneingabe zu einer gestarteten Session bereitstellen. Der Teilnehmer muss die Möglichkeit haben, sich mit seinem Namen einloggen zu können. Wenn keine Session gestartet ist, soll auf der Teilnehmer-Seite ein Texthinweis wie z.B “Bitte Warten” eingeblendet werden. Bei einer gestarteten Session steht als Überschrift die Frage der Session und das Eingabefeld wird angezeigt. Falls bereits von der moderierenden Person Kategorien erstellt wurden, sollen die erstellten Kategorien ebenfalls als ein Auswahlmenü (Dropdown-Liste) eingeblendet werden. Der Teilnehmer soll seine Ideen nach Kategorien zuordnen können. 
\\

Im Eingabe-beenden-Prozess wird die Frage der laufenden Session sowie das Eingabefeld und das Auswahlmenü von Kategorien ausgeblendet und stattdessen auf der Teilnehmer-Seite ein Texthinweis wie etwa “Bitte Warten” angezeigt. Der Benutzername und die Funktion zum Ausloggen soll in eine Navigationsleiste positioniert werden.

\subsubsection*{Präsentation-Seite}
Die Präsentation-Seite soll, wie bereits erwähnt, alle Eingaben aller Teilnehmer eines Workshops in Echtzeit präsentieren können. Die Frage der laufenden Session muss gut erkennbar dargestellt werden. Wenn die Session nicht läuft, wird der QR-Code zur Teilnahme am Workshop angezeigt.\\

Die Präsentation-Seite muss folgende Buttons beinhalten:
\begin{itemize}
\item Vollbildmodus-Button:
\begin{itemize}
\item passt die Seite im Vollbildmodus auf dem gesamten Bildschirm an.
\end{itemize}
\item QR-Code anzeigen-Button:
\begin{itemize}
\item blendet den QR-Code zur Teilnahme am Workshop ein.
\end{itemize}
\item QR-Code ausblenden-Button:
\begin{itemize}
\item nur sichtbar, wenn der QR-Code anzeigen-Button getätigt wird.
\item schaltet den angezeigten QR-Code wieder aus.
\end{itemize}
\item Kategorie erstellten-Button:
\begin{itemize}
\item nur sichtbar, nachdem der Eingabe beenden-Button auf der Controller-Seite getätigt wurde.
\item Kategorien für die Sortierung der Ideen werden erstellt. Jede Kategorie hat einen Titel. Der Moderator muss den Titel bearbeiten sowie die Kategorien löschen können.
\end{itemize}
\end{itemize}

\subsubsection*{Sortierung von Daten nach Kategorien}
Der Moderator kann die Daten auf der Präsentation-Seite nach Kategorien sortieren. Die Sortierung soll per Drag \& Drop\footnote{Ziehen und Ablegen} erfolgen. Die Daten, welche nicht sortiert wurden, sollen sich in der Kategorie “unsortiert” befinden. Die Kategorien selbst sollen nicht sortierbar sein. Nach dem Löschen einer nicht leeren Kategorie, müssen die darin befindlichen Daten automatisch nach Kategorie “unsortiert” geordnet werden.

\subsubsection*{Ergebnisse-Seite}
Nach Ausführen des Ergebnisse-Buttons auf der Controller-Seite eines ausgewählten Workshops soll die Ergebnisse-Seite alle Daten inklusive Kategorien von allen Sessions von diesem ausgewählten Workshop in Form einer Tabelle wiedergeben. Die Seite soll außerdem ein Button besitzen, über diesem der Moderator die Ergebnisse als eine PDF-Datei herunterladen kann.

\subsection{Nicht-funktionale Anforderungen}
\label{sec:nicht-funktionale anforderungen}
Im oberen Unterkapitel wurden die funktionalen Anforderungen aufgelistet. In diesem Kapitel werden die nicht-funktionalen Anforderungen formuliert, welchen zu diesem Projekt gehören sollen.

\subsubsection*{Layout, Handhabung und Benutzbarkeit}
Gemessen am Funktionsumfang sollte die zu entwickelnde Anwendung ein möglichst strukturiertes, einfaches und bedienerfreundliches Layout besitzen. Beim Entwurf und der Entwicklung der Anwendung sollten deshalb die folgenden Punkte beachtet werden:
\begin{itemize}
\item Die Verwendung der Webanwendung soll für Nutzer intuitiv sein. Der Nutzer soll mit wenigem Aufwand, ohne besondere Schulung und in kurzer Zeit durch die Webanwendung navigieren sowie sie verwenden und die wichtigen Funktionen der Webanwendung ausführen können.
\item Bereitstellung von Hilfeleistung in Form von Hilfetexten und Tooltips zur Förderung der intuitiven Bedienbarkeit.
\item Die Buttons sollen in unterschiedlichen Farben entsprechend der Funktionalität gestaltet werden.
\item Anzeigen von Bestätigungsdialogen beim Löschen von Workshops, Sessions und Kategorien sowie beim Beenden von Workshops.
\item Die Gestaltung der Webanwendung soll einheitlich nach vorgegebenen Designvorlagen vom Unternehmen erfolgen.
\end{itemize}

\subsubsection*{Plattformübergreifend}
Die Webanwendung soll unabhängig der Plattform funktionieren. Deshalb sollte die Webanwendung so gestaltet werden, dass das Layout der Webseite auf dem Computer, Tablet und Smartphone eine gleichbleibende Benutzerfreundlichkeit anbietet. Das bedeutet, die Inhalts- und Navigationselemente sowie der strukturelle Aufbau der Webanwendung sollten sich der Bildschirmauflösung aller Endgeräte anpassen. Somit ist es für den Nutzer möglich, diese Anwendung auf verschiedenen Endgeräten zu betreiben.

\subsubsection*{Browser und Betriebssysteme Unabhängigkeit}
Außer der Plattformunabhängigkeit soll die Anwendung in unterschiedlichen Browsern, wie Firefox oder Chrome und in unterschiedlichen Betriebssystemen genutzt werden können.

\subsubsection*{Anzeigen von Fehlermeldungen und Deaktivieren von Buttons bei nicht vorhandenen Verbindung zwischen Client und Server}
Bei nicht vorhandenen bzw. unterbrochenen Verbindungen zwischen Client und Server soll dem Nutzer einen Hinweistext bereitgestellt werden und folgende Buttons sollten dabei deaktiviert werden:
\begin{itemize}
\item Start session-Button
\item Ergebnisse-Button
\item Workshop Beenden-Button
\item Eingabe beenden- sowie Session beenden-Button
\end{itemize}
Die deaktivierten Buttons sollen bei wiederkehrender Verbindung automatisch reaktiviert werden.

\subsubsection*{Performance}
Die eingegeben Daten seitens der Teilnehmer sollen rechtzeitig und ohne Verzögerung auf der Präsentation-Seite geliefert werden.

\subsection{Muss- und Kann-Anforderungen}
Die funktionalen Anforderungen sowie nicht-funktionale Anforderungen wurden bereits in Unterkapitel \hyperref[sec:funktionale anforderungen]{\textbf{2.2.1}} und \hyperref[sec:nicht-funktionale anforderungen]{\textbf{2.2.2}} dargestellt. In diesem Kapitel werden die Muss- und Kann-Anforderungen formuliert. Die Muss-Anforderung wird mit Priorität “Hoch” gekennzeichnet, für die Kann-Anforderung wird die Priorität auf “Niedrig” gesetzt.\\

\begin{figure}[H]
  \centering  
  \includegraphics[scale=0.6]{img/Startseite.png}
  \caption{Muss- und Kann-Anforderungen für die Startseite (S)}
  \label{fig:startseite}
\end{figure}	

\begin{figure}[H]
  \centering  
  \includegraphics[scale=0.6]{img/Controller-Seite.png}
  \caption{Muss- und Kann-Anforderungen für die Workshop Controller-Seite (C)}
  \label{fig:controller-seite}
\end{figure}

\begin{figure}[H]
  \centering  
  \includegraphics[scale=0.6]{img/Teilnehmer-Seite.png}
  \caption{Muss- und Kann-Anforderungen für die Teilnehmer-Seite (T)}
  \label{fig:teilnehmer-seite}
\end{figure}

\begin{figure}[H]
  \centering  
  \includegraphics[scale=0.6]{img/Presentation-Seite.png}
  \caption{Muss- und Kann-Anforderungen für die Präsentation-Seite (P)}	
  \label{fig:presentation-seite}
\end{figure}

\begin{figure}[H]
  \centering  
  \includegraphics[scale=0.6]{img/Ergebnisse-Seite.png}
  \caption{Muss- und Kann-Anforderungen für die Ergebnisse-Seite (Er)}
  \label{fig:ergebnisse-seite}
\end{figure}

\begin{figure}[H]
  \centering  
  \includegraphics[scale=0.6]{img/Sonstige.png}
  \caption{Sonstige Muss- und Kann-Anforderungen für Webanwendung} 	
  \label{fig:sonstige}
\end{figure}

\subsection{Schematische Darstellung einer Bearbeitungshierarchie}
\begin{figure}[H]
  \centering  
  \includegraphics[scale=0.4]{img/Anforderung.jpeg}
  \caption{Darstellung der Bearbeitungshierarchie in der Webanwendung\newline [Quelle: eigene Abbildung]}
  \label{fig:anforderung}
\end{figure}











