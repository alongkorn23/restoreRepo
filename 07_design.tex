\chapter{Design}
\label{sec:design}
Aus den gesammelten Anforderungen vom \textbf{Kapitel \ref{sec:analyse}} wird es in diesem Kapitel um den Entwurf der Weboberfläche (WebUI) gehen. Die grundlegende Gestaltungsrichtlinie darunter die einheitliche Verwendung von Icons, Farben und Schriftgestaltung werden dabei beschrieben. Außerdem wird der Aufbau der zu entwickelnden Webanwendung mithilfe von Mockups dargestellt.\bigskip

\par
\begingroup
\leftskip=4em % Parameter anpassen
\rightskip\leftskip
\noindent \glqq Gutes Design ist so wenig Design wie möglich
Weniger Design ist mehr, konzentriert es sich doch auf das Wesentliche, statt die Produkte mit Überflüssigem zu befrachten. Zurück zum Puren, zum Einfachen!\grqq{} - Dieter Rams (vgl. 10 Thesen für gutes Design\footnote{\url{https://www.vitsoe.com/de/ueber-vitsoe/gutes-design}})
\par
\endgroup
\bigskip

\section{Gestaltungsrichtlinie}
\label{sec:gestaltungsrichtlinie}
Um der Webanwendung ein einheitliches, strukturiertes und benutzerfreundliches Design zu geben ist es erforderlich, feste Layoutvorgaben zu definieren. Sie sorgt für eine verständliche und intuitiv bedienbare Benutzeroberfläche, sodass eine positive Empfindung der Nutzer bei der Bedienung der Webanwendung hervorgerufen wird und ohne größere Einarbeitungszeit beherrscht werden kann. Diese Vorgabe wird auf alle Seiten der Webanwendung angewendet.

\subsection{Farben}
\label{subsec:farben}
Um die Webanwendung übersichtlich zu halten, wird auf die Verwendung mehrerer Farben verzichtet. Es werden grundsätzlich die Farben grau und weiß verwendet. Die Gestaltung von Buttons und Daten werden wiederum mit bunten Farben gestaltet, damit sie optisch auffallend sind.

\subsection{Schriftgestaltung}
\label{subsec:schriftgestaltung}
Bei der Schriftgestaltung, wie Schriftgröße und Schriftart, ist darauf zu achten, dass sie eine gute Lesbarkeit und ein modernes Aussehen bieten. Deshalb wird eine Schriftart verwendet, bei der es sich um eine serifenlose Schrift handelt.

\subsection{Icons}
\label{subsec:icons}
Die Buttons werden mit Icons gestaltet, um Inhalte schneller zu verstehen und den Nutzen der Funktionen zu verdeutlichen. Hierbei handelt es sich um allgemein übliche und bei mobilen Anwendungen bekannte Icons. Sie sind so gewählt, dass sich der Nutzer bei den grundlegenden Funktionen auf der Webanwendung schnell zurechtfindet. Bei den Icons handelt es sich hierbei um keine Grafiken, sondern um eine sogenannte Icon Fonts, welche über eigenes Stylesheet geladen werden. Die Fonts haben vor allem den Vorteil, dass sie auf jede Bildschirmgröße skaliert werden können. Für diese vorliegende Arbeit werden die Icon Fonts von Font Awesome\footnote{vgl. \url{https://fontawesome.com/}} verwendet.\bigskip

\begin{figure}[H]
  \begin{center}
    \includegraphics[scale=0.5]{img/icons}
	\caption{Verwendete Icon Fonts} 
	\footnotesize\sffamily\textbf{Quelle:} \url{https://fontawesome.com/icons?d=gallery}  
	\label{fig:icons}
  \end{center}   
\end{figure}

\newpage
\section{Konzeption}
\label{sec:konzeption}
In diesem Abschnitt werden die erstellten Mockups, also die Wireframes der verschiedenen Seiten dargestellt und erläutert. Die definierte Gestaltungsrichtlinie sollte hierbei bei der Konzeption eingehalten werden.

\subsection{Mockup der Willkommensseite}
\label{subsec:mockup der willkommensseite}
Wenn die richtige URL aufgerufen wird, erscheint die Willkommensseite mit der Aufforderung zur Eingabe von Benutzernamen und Passwort \textbf{(Abbildung \ref{fig:mockup für die anmeldung})}. Zur Erinnerung, die Benutzerdaten werden in dieser Arbeit manuell in der Datenbank angelegt. Der Registrierungsvorgang wird in dieser Arbeit nicht vorhanden sein. Dieser wird für die zukünftige Weiterentwicklung festgehalten.\bigskip

Die Formulardaten werden über den Anmelde-Button verschickt, welcher ein Ereignis (Event) auslöst, um den richtigen User zu authentifizieren. Falls der User nicht existiert, wird eine Fehlermeldung auf der Seite ausgegeben \textbf{(Abbildung \ref{fig:mockup für die anmeldungsfehler})}.\bigskip

\begin{figure}[H]
	\centering
  \begin{minipage}[t]{0.45\linewidth}
  	    \includegraphics[width=.75\linewidth]{img/willkommenseite1}
		\caption{Mockup für die Anmeldung}
		\label{fig:mockup für die anmeldung}
  \end{minipage}
\hfill
  \begin{minipage}[t]{0.45\linewidth}
    	    \includegraphics[width=.75\linewidth]{img/willkommenseite2}
		\caption{Mockup für den Anmeldungsfehler}
		\label{fig:mockup für die anmeldungsfehler}
  \end{minipage}
\end{figure}

\newpage
\subsection{Mockup der Hauptseite}
\label{subsec:mockup der hauptseite}
Nach der erfolgreichen Anmeldung wird der Nutzer, der für die Durchführung des Workshops verantwortlich ist, auf die Hauptseite weitergeleitet \textbf{(Abbildung 4.3)}. Auf dieser Seite sind drei grundlegende Bereiche (Nr.1, Nr.2 und Nr.3) zu sehen:\bigskip

\begin{figure}[H]
  \begin{center}
    \includegraphics[scale=0.45]{img/hauptseite}
	\caption{Mockup der Hauptseite}  
	\label{fig:mockup für die hauptseite}
  \end{center}   
\end{figure}

\begin{enumerate}
\item Header:\\
Im Header befindet sich die Navigationsleiste. In dieser sind drei Navigationselemente enthalten.
\begin{itemize}
\item Workshoppy: \\
Navigiert den Nutzer zur Hauptseite. 
\item Profil:\\
Die angemeldete Person kann unter dem Profil seine Daten verwalten, wie z.B. seine Accountdaten (Benutzername, Passwort) ändern, den Account löschen und seine hinterlegten Personendaten anzeigen lassen.
\item Ausloggen:\\
Ermöglicht dem Nutzer, sich ordnungsgemäß von der Webanwendung auszuloggen.
\end{itemize}
\item Workshop-Liste:\\
In diesem Bereich werden die erstellten Workshops aufgelistet. Mit dem \glqq Workshop erstellen\grqq{}-Button kann ein neuer Workshop erstellt werden. Durch das Anklicken des Edit- sowie Löschen-Buttons kann der Workshop gezielt bearbeitet und gelöscht werden.\bigskip

Die \textbf{Abbildung \ref{fig:workshop erstellen}} zeigt das Erstellen eines neuen Workshops. Der Titel ist ein Pflichtfeld und muss beim Erstellen angegeben werden. Als Option steht ein Textbereich für die Agenda zur Verfügung.\bigskip

\begin{figure}[H]
  \begin{center}
    \includegraphics[scale=0.45]{img/workshop_erstellen}
	\caption{Mockup für das Erstellen eines neuen Workshops}  
	\label{fig:workshop erstellen}
  \end{center}   
\end{figure}

Beim Titel des Workshops handelt es sich um ein Linktext. Beim Anklicken wird der Moderator zur \textbf{Controller-Seite} dieses Workshops geführt.
\item Beendete Workshops:\\
Die beendeten Workshops werden in diesem Bereich archiviert. Der Ergebnisse-Button führt zur \textbf{Ergebnisse-Seite} des archivierten Workshops.
\end{enumerate}

\newpage
\subsection{Mockup der Controller-Seite}
\label{subsec:mockup der controller-seite}
Jeder Workshop hat seine eigene Controller-Seite. Die \textbf{Abbildung \ref{fig:controller-seite}} zeigt beispielsweise die Controller-Seite von \glqq Workshops1\grqq{}. Auf dieser sind der Titel des Workshops (Nr.1), die Navigation-Tabs (Nr.2) und die Session-Liste (Nr.3) zu sehen.\bigskip

\begin{figure}[H]
  \begin{center}
    \includegraphics[scale=0.45]{img/controllerseite}
	\caption{Mockup der Controller-Seite}  
	\label{fig:controller-seite}
  \end{center}   
\end{figure}

Es sind drei Navigations-Tabs (Nr.2) vorhanden:
\begin{enumerate}
\item Das Navigation-Tab \glqq WS-Controller\grqq{} beinhaltet vier folgende Buttons:
\begin{itemize}
\item Client-Button:\\
öffnet die \textbf{Teilnehmer-Seite} als neues Browser-Tab. Auf dieser Seite können die Teilnehmer die Dateneingabe tätigen.\\
\textbf{Anmerkung:} Der Client-Button wird in der zukünftigen Weiterentwicklung entfernt, da der Moderator nicht für die Dateneingabe beteiligt werden darf. Für diese Arbeit wird der Client-Button aufgrund des Funktionstests erstmal erhalten bleiben.
\item Präsentation-Button:\\
öffnet als neues Browser-Fenster die \textbf{Präsentation-Seite}. Mittels Beamer präsentiert sie den Teilnehmern die eingegebenen Daten in Echtzeit.
\item Ergebnisse-Button:\\
öffnet ein neues Browser-Tab und ruft die \textbf{Ergebnisse-Seite} auf. Die Ergebnisse des Workshops werden dargestellt. Der Ergebnisse-Button ist erst aktiviert, wenn die Ergebnisse vorhanden sind.
\item Beenden-Button:\\
beendet den laufenden Workshop und leitet den Moderator zur \textbf{Hauptseite} weiter. Der Workshop wird anschließend in \glqq Beendete Workshops\grqq{} archiviert \textbf{(Abbildung \ref{fig:mockup für die hauptseite})}.
\end{itemize}
\item Die Agenda, falls sie vorhanden ist, wird im Navigation-Tab \glqq Agenda\grqq{} dargestellt.
\item Neben dem Einscannen des QR-Codes auf der Präsentation-Seite \textbf{(Abbildung \ref{fig:mockup für das anzeigen des qr-codes})} können die Teilnehmer im Navigation-Tab \glqq Teilnehmer\grqq{} die Einladung per Mail senden lassen, um am Workshop teilzunehmen.\bigskip

\begin{figure}[H]
  \begin{center}
    \includegraphics[scale=0.45]{img/einladungsmail}
	\caption{Mockup für das Navigation-Tab \glqq Teilnehmer\grqq{}}  
	\label{fig:mockup für einladungsmail}
  \end{center}   
\end{figure}
\end{enumerate}

Das Brainstorming wird in der Session-Liste (Nr.3) in \textbf{Abbildung \ref{fig:controller-seite}} durchgeführt. Zunächst muss der Moderator mit dem \glqq Session Erstellen\grqq{}-Button eine neue Session anlegen \textbf{(Abbildung \ref{fig:mockup für das erstellen einer neuen session})}.\bigskip

\begin{figure}[H]
  \begin{center}
    \includegraphics[scale=0.45]{img/session_erstellen}
	\caption{Mockup für das Erstellen einer neuen Session}  
	\label{fig:mockup für das erstellen einer neuen session}
  \end{center}   
\end{figure}

Die behandelte Frage, die auf der \textbf{Teilnehmer- und Präsentation-Seite} zu sehen sein wird, muss definiert werden. Als Option kann der Titel der Session angegeben werden. Wie viele Sessions in einem Workshop benötigt werden, das entscheidet der Moderator selbst. Er kann unbegrenzt viele Sessions erstellen.

\newpage
Neben jeder Session sind in \textbf{Abbildung \ref{fig:controller-seite}} drei Buttons zu sehen. 
\begin{enumerate}
\item Edit-Button:\\
Mit diesem Button kann die Titel- sowie Fragenänderung durchgeführt werden.
\item Löschen-Button:\\
Der Löschen-Button löscht die Session inklusive ihrer zugehörigen Daten.
\item Starten-Button:\\
Es wird erst \glqq gebrainstormt\grqq{}, wenn die Session gestartet ist. Während die Session läuft, darf sie nicht bearbeitet und gelöscht werden. Alle Buttons von nicht aktiven Sessions werden auch in dieser Phase deaktiviert. Es kann nur eine Session gestartet werden. Außerdem kann der Workshop bei laufender Session nicht beendet werden. Der \glqq Beenden\grqq{}-Button in WS-Controller \textbf{(Abbildung \ref{fig:mockup für die aktive session})} wird deshalb deaktiviert.\bigskip

Es gibt zusätzlich noch zwei weiteren Buttons, welche erst sichtbar werden, wenn eine Session gerade läuft. Das ist der \glqq Eingabe beenden\grqq{}- und \glqq Session Beenden\grqq{}-Button \textbf{(Abbildung \ref{fig:mockup für die aktive session})}.

\begin{figure}[H]
  \begin{center}
    \includegraphics[scale=0.45]{img/session_ist_gestartet}
	\caption{Mockup für die aktive Session}  
	\label{fig:mockup für die aktive session}
  \end{center}   
\end{figure}

Der \glqq Eingabe beenden\grqq{}-Button bricht die Eingabefunktion auf der Teilnehmer-Seite \textbf{(Abschnitt \ref{subsec:Mockup der Teilnehmer-Seite})} ab. Demzufolge können die Teilnehmer keine weiteren Daten mehr eingeben. Der \glqq Session Beenden\grqq{}-Button beendet die gerade laufende Session. Auf der \textbf{Teilnehmer-Seite} wird durch den Klick auf dem \glqq Session Beenden\grqq{}-Button der Infotext \glqq Bitte Warten\grqq{} angezeigt und gleichzeitig wird der QR-Code auf der \textbf{Präsentation-Seite} dargestellt \textbf{(Abbildung \ref{fig:mockup für das anzeigen des qr-codes})}. Erst nach dem Beenden einer laufenden Session werden alle zuvor deaktivierten Buttons wieder reaktiviert.
\end{enumerate}

\subsection{Mockup der Teilnehmer-Seite}
\label{subsec:Mockup der Teilnehmer-Seite}
Um auf diese Seite zu kommen, müssen die Teilnehmer den QR-Code entweder über Ihre Mobilgeräte auf der Präsentation-Seite \textbf{(Abschnitt \ref{subsec:mockup der präsentation-seite})} einscannen oder sie lassen sich per Mail die Einladung zur Teilnahme am Workshop zusenden. Die Teilnehmer-Seite stellt jedem Workshop-Teilnehmer die Dateneingabefunktion zu einer gestarteten Session bereit. Beim Aufrufen der Seite werden die Teilnehmer zunächst aufgefordert, ihren Benutzernamen einzugeben \textbf{(Abbildung \ref{fig:mockup für eingabe der benutzernamen})}.

\begin{figure}[H]
  \begin{center}
    \includegraphics[scale=0.45]{img/teilnehmerseite}
	\caption{Mockup für Eingabe der Benutzernamen}  
	\label{fig:mockup für eingabe der benutzernamen}
  \end{center}   
\end{figure}

Nach Eingabe ihres Benutzernamens werden die Teilnehmer auf die Eingabefunktion weitergeleitet. Die \textbf{Abbildung \ref{fig:mockup für die anzeige der infotext}} beschreibt, dass die Teilnehmer-Seite gerade auf das Kommando des Moderators wartet. Sobald er eine Session startet, werden die Teilnehmer für die Funktionen zur Dateneingabe freigeschaltet \textbf{(Abbildung \ref{fig:mockup für die eingabefunktion})}.

\begin{figure}[H]
	\centering
  \begin{minipage}[t]{0.45\linewidth}
  	    \includegraphics[width=.75\linewidth]{img/teilnehmerseite2}
		\caption{Mockup für die Anzeige der Infotext}
		\label{fig:mockup für die anzeige der infotext}
  \end{minipage}
\hfill
  \begin{minipage}[t]{0.45\linewidth}
    	    \includegraphics[width=.8\linewidth]{img/teilnehmerseite3}
		\caption{Mockup für die Eingabefunktion}
		\label{fig:mockup für die eingabefunktion}
  \end{minipage}
\end{figure}

In der Navigationsleiste in \textbf{Abbildung \ref{fig:mockup für die anzeige der infotext}} sowie in \textbf{Abbildung \ref{fig:mockup für die eingabefunktion}} sind zwei Navigationselemente zu sehen. Der Benutzername ist der Name des Teilnehmers, welcher zuvor eingegeben wurde. Das Ausloggen erlaubt dem Teilnehmer, seinen Benutzernamen zu ändern. Die Session-Frage in \textbf{Abbildung \ref{fig:mockup für die eingabefunktion}} ist die eigentliche Frage, welche in dieser Session behandelt wird. Die Textarea dient der Dateneingabe. Beim Klick auf den Abschicken-Button wird die Dateneingabe in Echzeit auf der \textbf{Präsentation-Seite} dargestellt.

\subsection{Mockup der Präsentation-Seite}
\label{subsec:mockup der präsentation-seite}
Die Präsentation-Seite dient der Darstellung der Dateneingabe von allen Teilnehmern in Echtzeit. Die Seite hat zwei Zustände, nämlich passiv und aktiv.

\begin{itemize}
\item passiver Zustand:\\
Dieser Zustand bedeutet, dass momentan keine Session läuft. Die Präsentation-Seite zeigt bei diesem Zustand den QR-Code zur Teilnahme am Workshop an. Bevor die Session tatsächlich beginnt, können die Teilnehmer den QR-Code über Ihre Mobilgeräte einscannen, um am Workshop teilzunehmen \textbf{(Abbildung \ref{fig:mockup für das anzeigen des qr-codes})}. Falls es eine Agenda zu dem Workshop gibt, wird sie neben dem QR-Code dargestellt \textbf{(Abbildung \ref{fig:mockup für das anzeigen eines qr-code und einer agenda})}.\bigskip

\begin{figure}[H]
	\centering
  \begin{minipage}[t]{0.45\linewidth}
  	    \includegraphics[width=.8\linewidth]{img/präsentationseite1}
		\caption{Mockup für das Anzeigen eines QR-Codes}
		\label{fig:mockup für das anzeigen des qr-codes}
  \end{minipage}
\hfill
  \begin{minipage}[t]{0.45\linewidth}
    	    \includegraphics[width=.8\linewidth]{img/präsentationseite2}
		\caption{Mockup für das Anzeigen eines QR-Code und einer Agenda}
		\label{fig:mockup für das anzeigen eines qr-code und einer agenda}
  \end{minipage}
\end{figure}

\newpage
\item aktiver Zustand:\\
Die Präsentation-Seite befindet sich im aktiven Zustand, wenn die Session gestartet ist. Auf der Präsentation-Seite werden die Dateneingaben der Teilnehmer in Echtzeit präsentiert \textbf{(Abbildung \ref{fig:mockup für die darstellung der dateneingabe auf der präsentation-seite})}.\bigskip

\begin{figure}[H]
  \begin{center}
    \includegraphics[scale=0.5]{img/präsentationseite3}
	\caption{Mockup für die Darstellung der Dateneingabe auf der Präsentation-Seite}  
	\label{fig:mockup für die darstellung der dateneingabe auf der präsentation-seite}
  \end{center}   
\end{figure}

Die behandelte Frage (Nr.1) ist am oberen Inhaltsbereich zu sehen. Die Eingaben der Teilnehmer (Nr.2) werden wie ein Notizzettel visualisiert. Ein Notizzettel besteht aus zwei Teilen, dem Namen des Teilnehmers und seine Idee. Am unteren Bereich der Präsentation-Seite befinden sich zwei Buttons (Nr.3). Während die Session läuft, kann der QR-Code des Workshops mit dem Drücken des QR-Code-Buttons angezeigt werden. Wenn der QR-Code-Button getätigt wird, wandelt er sich anschließend im \glqq QR-Code ausblenden\grqq{}-Button um. Die Session wird dabei nicht beendet und kann mit dem Drücken des \glqq QR-Code ausblenden\grqq{}-Button wieder auf dem Zustand kommen, wie auf der \textbf{Abbildung \ref{fig:mockup für die darstellung der dateneingabe auf der präsentation-seite}} dargestellt ist. Der QR-Code kann sowohl im passiven Zustand \textbf{(Abbildung \ref{fig:mockup für das anzeigen des qr-codes})} als auch im aktiven Zustand dargestellt werden. Der Vollbild-Button wandelt die Präsentation-Seite in den Vollbildmodus um.\bigskip

Um die eingegebenen Daten auf der Präsentation-Seite zusammenfassen zu können, muss die Ideensammlungsphase beendet werden. Dafür klickt der Moderator auf den \glqq Eingabe beenden\grqq{}-Button auf der Controller-Seite, wie in \textbf{Abbildung \ref{fig:mockup für die aktive session}} zu sehen ist. Daraus folgt, dass auf der \textbf{Teilnehmer-Seite} die Eingabefunktion ausgeblendet und stattdessen der Infotext \glqq Bitte Warten\grqq{} angezeigt wird. Durch den Klick auf den \glqq Eingabe beenden\grqq{}- Button auf der \textbf{Controller-Seite} wird der \glqq Kategorie erstellen\grqq{}-Button auf der Präsentation-Seite freigeschaltet, mit dem der Moderator Kategorien erstellen kann \textbf{(Abbildung \ref{fig:mockup für zusammenfassung-modus auf der präsentation-seite})}.

\begin{figure}[H]
  \begin{center}
    \includegraphics[scale=0.5]{img/präsentationseite4}
	\caption{Mockup für das Zusammenfassen von Daten auf der Präsentation-Seite}  
	\label{fig:mockup für zusammenfassung-modus auf der präsentation-seite}
  \end{center}   
\end{figure}

Direkt auf der Präsentation-Seite kann der Moderator die Daten mit der Drag-\&-Drop-Funktion in Kategorien zusammenfassen. Die Kategorien selbst lassen sich nicht verschieben. Das Löschen einer Kategorie erfolgt mit den Klick auf dem X-Button. Es wird nur die Kategorie gelöscht, d.h. die darin befindlichen Daten bleiben auf der Präsentation-Seite erhalten.
\end{itemize}

\subsection{Mockup der Ergebnisse-Seite}
\label{subsec:mockup der ergebnisse-seite}
Mit Klick auf den Ergebnisse-Button in \textbf{Abbildung \ref{fig:mockup für die hauptseite}} sowie in \textbf{Abbildung \ref{fig:controller-seite}} wird die Ergebnisse-Seite des Workshops als neues Browser-Tab geöffnet. Auf dieser Seite befinden sich die Daten inklusive Kategorien des Workshops. Die Ergebnisse-Seite beinhaltet außerdem einen Button, mit dem die Ergebnisse als eine PDF-Datei heruntergeladen werden können \textbf{(Abbildung \ref{fig:mockup für die darstellung der ergebnisse des workshops})}.

\begin{figure}[H]
  \begin{center}
    \includegraphics[scale=0.45]{img/ergebnisse_seite}
	\caption{Mockup für die Darstellung der Ergebnisse des Workshops}  
	\label{fig:mockup für die darstellung der ergebnisse des workshops}
  \end{center}   
\end{figure}

\subsection{Zusammenfassung der Konzeption}
\label{subsec:zusammenfassung der konzeption}

\begin{figure}[H]
  \begin{center}
    \includegraphics[scale=0.45]{img/Anforderung_neu}
	\caption{Zusammenfassung der Konzeption der zu entwickelnden Webanwendung}  
	\label{fig:zusammenfassung der konzeption}
  \end{center}   
\end{figure}


